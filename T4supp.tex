\documentclass[T4paper.tex]{subfiles}
\begin{suppinfo}
\section{Experimental}
\subsection{Discrete Conformations and the Ligands}
T4 L99A contains an engineered apolar cavity which is our binding site of interest.
The 8 congeneric ligands are all apolar and begins with a simple benzene ring.
Subsequent ligands are simply addition of a methyl group to generate a growing tail up until n-hexylbenzene.
In response to the growing ligand, the crystal structures show the protein will adopt into 3 conformations aptly named: closed, intermediate, and open.
Primarily, the motion of the protein occurs in the F-helix (residues 107-115), which serves as a sort of gating mechanism into the apolar cavity.
As the ligand tail expands, the F-helix transitions from closed to open, exposing the cavity to the bulk solvent.
From observed electron densities of the F-helix in \cite{Merski2015}-Fig2, the ligands occupy each of the conformations given in Table: ~\ref{tbl:loop-occ}

From a protein conformation clustering analysis, shown \cite{Merski2015}, Val111 occupies 3 distinct states iin accordance to the 3 protein conformations.
Visualization of side-chain Val111 from the x-stal structures, shows the backbone alpha-carbon moving ~1.25\AA when transitioning from closed to intermediate, ~3.25\AA intermediate to open, and ~3.50\AA closed to open.
From our MD simulations, it is primarily the repulsive interactions between the ligands and Val111 that drive the F-helix to the open state.

\subsection{Ligand Binding Affinities}
Table ~\ref{tbl:lig-aff}
Details of the ligand binding affinities and how they were obtained can be found in the paper \cite{Merski2015} \\
***Section "Energy of Ligand Binding and Conformational Strain"***\\
Ligands n-pentylbenzene and n-hexylbenzene affinities are inaccessible due to solubility limits.
Generally, as the ligands grow from benzene to n-butylbenzene, the affinity rises linearly.

\section{FEP+ vs. LigandFEP}
   \begin{itemize}
   \item Comparisons between the two protocols will be used to show that, although limited, LigandFEP does not result in a difference in performance of accurate predicted relative binding free energies.
   \item Comparisons between FFs will show that the new and improved OPLS3 parameters are give better RBFE predictions for larger transformations.
   \item ***INSERT TABLE COMPARING OPLS2005/OPLS3 BETWEEN TWO PROTOCOLS***
      \\ Both are within the same MUE/RMSE range
   \item Highlight inconsistency in predicted RBFE when the protein starting conformation is varied.
   \item Predicted ddGs are in the wrong direction frequently when starting from protein closed.
      If we assume smaller to larger ligands should yield favorable (-) ddGs.
   \end{itemize}

\section{OPLS2005 vs. OPLS3}

\section{Case studies}
\subsection{Small Ligands}
   \begin{itemize}
   \item Small transformations and ligands generally occupy protein closed state.
   \item Equal performance with either FF.
   \item Highlight good case: Toluene to Ethylbenzene with OPLS3
      \begin{itemize}
      \item Small perturbation: Adding one carbon
      \item Both ligands occupy the closed state with some intermediate
      \item Protein starting conformation does not result in a large discrepancy in predicted ddGs.
      \item ***INSERT RMSD GRAPH***
      \item RMSD graph over the lambda=0 and lambda=1 corresponding to the toluene and ethylbenzene end states shows at each time point which conformation (reference to crystal structure) the simulation has the lowest RMSD to.
         \\ Purple = Closed, Teal = Intermediate, and Green = Open.
      \item Highlight good sampling/number of transitions between either state when starting from closed.
      \item Highlight points ~0-1ns are still stuck in open conformation. 240ps relaxation protocol wasn't sufficient to discard few open points.
         \\ Show that this does not have a large impact in the final ddG from sliding time.
      \end{itemize}

   \item Highlight not-so-good case: Toluene to n-propylbenzene
      \begin{itemize}
      \item Perturbation involves adding two carbons
      \item Discrepancy from protein starting conformation increases.
      \item Highlight more open points are present (0-2ns) in open runs.
      \end{itemize}
   \item Growing ligands require more time for helix to relax out of open state.
   \item Inclusion of protein residues in REST region should allow for faster transition between states which will give us a lower discrepancy between open/closed runs.
   \item Apply pREST to previous case and show faster relaxation out of open conformation resulting in a lower discrepancy.
   \item ***INSERT COMPARISON BETWEEN NORMAL AND PREST RUNS WITH EXPERIMENTAL ddG ***
   \item Show pREST increases agreement with experiment and lowers error from protein starting conformation.
   \end{itemize}

\subsection{Intermediate Ligands}
***Insert all data***
\subsection{Open Ligands}
Simulations starting from closed give a predicted ddG of +2.74 kcal/mol and from open +1.37 kcal/mol.
Upon closer inspection of the closed simulations corresponding to the final state of n-hexylbenzene, we find that the protein does not remain trapped in its initial state.
Instead, the region of the helix around Gly113 briefly opens to relieve the strain but quickly closes back.
Hence, we still see some strain energy as the protein fails to completely stabilize into the open conformation.
Viewing the open simulations, reveals that the protein is no longer trapped only in the open state and makes transitions into the intermediate and closed states.
In making these transitions, the protein also experiences strain as the tail pushes the F-helix back into open from the other conformational states.\\

***INSERT RMSD GRAPH SHOWING TRANSITIONS BETWEEN STATES FOR BENZENE TO NHEXYL***\\
***By comparing these graphs we show that inclusion of protein residues in the REST region does allow for more/faster transitions between states***\\
***Compare pREST runs with default REST in order to show that pREST allows for the protein to get of being trapped in its initial state.\\

Here we have shown pREST allows for more motion in the helix in comparison to the default REST protocol.
Although there is a reduction in the discrepancy, it is still fairly large (+1.37 kcal/mol).
Thus, we cannot say these are converged, where this poor convergence results from inadequate sampling of all the protein conformational states.
Such is the case of closed pREST simulations of n-hexylbenzene, where we only see a partial opening of the helix.
Similarly, in open pREST simulations with n-hexylbenzene, the helix does not enter the closed conformation.
Where as it is expected to at least partially occupy the closed state, according to the loop/ligand occupancy table.
It seems likely that the overall motion of the helix is not entirely accessible in the range of up to 50ns (longest we have simulated).\\

***INSERT DATA FROM 55ns pREST simulations***\\
***Table showing all closed to open cases***\\
***Here we can show that the discrepancy does not entirely go away, but gradually gets smaller with smaller perturbations***\\
***Show that in 50ns the protein is only barely able to stabilize in the open state and vice versa***\\
\end{suppinfo}