\documentclass[journal=jctcce,manuscript=article]{achemso}

\usepackage[version=3]{mhchem} % Formula subscripts using \ce{}
\usepackage[T1]{fontenc}       % Use modern font encodings
\usepackage{graphicx}
\usepackage[table,xcdraw]{xcolor}
\usepackage{subfiles}

\author{Nathan Lim}
\email{limn1@uci.edu}
\affiliation[University of California, Irvine]
{Department of Pharmaceutical Sciences}

\title{Sensitivity in predicted relative binding free energies from incremental ligand changes within a model binding site}

\begin{document}

\begin{abstract}
Despite innovations in sampling techniques for molecular dynamics (MD), reliable prediction of protein-ligand binding free energies from MD remains a challenging problem, even in well studied model binding sites like the apolar cavity of T4 Lysozyme L99A \cite{Boyce2009}.
In this study, we model recent experimental results that show the progressive opening of the binding pocket in response to a series of homologous ligands \cite{Merski2015}.
Even while using enhanced sampling techniques, we demonstrate that the predicted relative binding free energies (RBFE) are still highly sensitive to the initial protein conformational state.
Particularly, we highlight the importance of sufficient sampling of protein conformational changes and possible techniques for addressing the issue.
\end{abstract}

\pagebreak

\section{Introduction}
%Jotted down ideas only
Medicinal chemistry programs typically focus on changes in ligand binding affinity from incremental changes to the ligand.
Focus on how the protein adapts to the changes in the ligand is generally neglected.
T4 L99A is well studied experimentally and computationally.
It is frequently used as a model binding site in free energy prediction studies.
In this study, 8 congeneric ligands were investigated, where addition of a single methyl group was used to lengthen the ligand.
Through determination of protein-ligand bound x-ray crystal structures it was revealed T4 Lysozyme adopts 3 discrete conformations in response the series of growing ligands.
Consideration of the protein adaptations into discrete conformations may be an important aspect in inhibitor design.

\section{Methods}
\subsection*{FEP protocols}
Using the Schr\"{o}dinger application suite (release 2015-3), two FEP protocols were used in this project: FEP+\cite{FEPplus} and LigandFEP\cite{LigandFEP}.
FEP+ is a fully automated work flow that plans perturbation pathways based off the LOMAP \cite{LOMAP} mapping algorithm which uses the maximum common substructure (MCS) between any pair of compounds.
LigandFEP is an academic toolkit that generates the configuration files to perform the free energy calculation but is limited in the sense that the user must plan each perturbation path instead.
Both FEP protocols use the default Desmond relaxation protocol and the FEP/REST methodology \cite{REST,REST2,FEP/REST,FEP/RESTapp}.
Results and discussion sections will present information from using the LigandFEP protocol, additional data not discussed here can be found in the supplementary info.

\subsection*{Protein/Ligand preparation}
Protein structures were taken from PDBs: 4W52,4W53,4W54,4W55,4W56,4W57,4W58,4W59 corresponding to bound structures of benzene, toluene, ethylbenzene, n-propylbenzene, sec-butylbenzene, n-butylbenzene, n-pentylbenzene, and n-hexylbenzene, respectively\cite{Merski2015}.
Each simulation will start from either the protein closed state (PDB:4W52) or the open state (PDF:4W59).
When using the FEP+ protocol, ligand crystal structure positions were used as the starting position of the simulation.
When using the LigandFEP protocol, a similar workflow to the tutorial\cite{LigandFEP} was followed.

Generally, two options were taken:\\
(1a) If the simulation starts from the protein closed state, the benzene crystal position was used as a reference for fragment building (PDB:4W52).\\
(1b) The corresponding ligand in the transformation was built by duplicating benzene in place and adding methyl groups.\\
(2a) If the simulation starts from the protein open state, the n-hexylbenzene crystal position was used as reference for fragment building (PDF:4W59).\\
(2b) The corresponding ligand in the transformation was built by duplicating n-hexylbenzene in place and deleting methyl groups.\\

Ligand tail fragments were added using the Build/Fragments toolbar in Maestro \cite{Maestro} and were not overlaid or docked. 
As the ligand tails were built, bonds were manually rotated so that the tail was oriented in a similar manner as in their corresponding crystal structure.
Following, the newly added atoms in the tail were locally minimized, leaving the core in its initial position.
This was done in an attempt to minimize the core RMSD, which LigandFEP uses to select the ligand heavy atoms to include in the REST region. 

All proteins were prepared and aligned in Maestro using the `Protein Preparation Wizard'\cite{ProteinPrepWizSoftware,Epik,Impact,Prime,ProteinPrepWizPaper} tool and with the following settings enabled (as they appear in the Maestro menu):
   \begin{itemize}
   \item Preprocess: Assign bond orders, Add hydrogens, Create zero-order bonds to metals, Create disulfide bonds, Cap termini, Delete waters beyond 5\AA from het groups
   \item Refine: Sample water orientations, Use PROPKA pH: 7.0, Remove waters with less than 3 H-bonds to non-waters, and restrained minimization.
   \end{itemize}

\subsection*{Simulation details}
Desmond\cite{DESMONDSoftware,DESMONDarticle,DESMONDPaper1,DESMONDPaper2} simulation protocols have been described previously in the supporting information\cite{FEPplus} or can be found in greater detail in the Desmond User Manual\cite{DESMONDManual}.
In summary, solute molecules are restrained to their initial positions while minimizing using a Brownian dynamics NVT integrator for 100ps, followed by 12ps simulations at 10K with a NVT ensemble and then a NPT ensemble using the Langevin method \cite{Langevin}.
Next, a 24ps simulation followed by a final 240ps simulation with solute molecules unrestrained, both are carried at room temperature with a NPT ensemble using Langevin.
Production simulations were carried out to a length of up to 55ns for closed-open transformations and up to 25ns for closed-intermediate transformations.
FEP/REST simulations were run on four GeForce GTX Titan Black GPUs using the Desmond/GPU engine with OPLS2005\cite{OPLS2005} and OPLS3\cite{OPLS3} forcefield parameters.
Calculated free energies were determined using the Bennett acceptance ratio\cite{BAR} (BAR) with error estimations using both bootstrapping and BAR analytical error prediction\cite{BARerror}.
Hysteresis around closed thermodynamic cycles and best estimates of the free energies with their errors were calculated using the cycle closure algorithm discussed in a previous publication\cite{FEP/REST}.

\subsection*{REST region selection}
In this study, by default, only heavy atoms in the ligand were included in the REST region unless specified otherwise.
Further details on the temperature profile and how the REST region is normally selected can be found in previous studies\cite{FEP/REST,FEPplus} in the supporting information.
Simulations that included protein heavy atoms in the REST region are referred to with the `pREST' label, where selection of the particular residues is described as follows.
Based on visual inspection of our molecular dynamics simulations and considering the F-helix spans residues 107-115, we selected residues Glu108, Val111, and Gly113 to include into the REST region (Fig~\ref{fig:C2O}).
Glu108 sits near the start of the helix which appears as a hinge point for the opening and closing of the binding cavity (Fig~\ref{fig:Glu108-C2O}).
Following, Val111 appears in the middle of the helix and was observed to undergo the largest motion during protein conformational changes (Fig~\ref{fig:Val111-C2O}).
Gly113 was included in order to collectively have hot regions approximately at the start, middle and end points of the helix.

\subsection*{RMSD analysis}
%DLM: Methods seems to end abruptly. Other things to maybe include:
% How was analysis done
% How you did your RMSD calculations/colorcoding
% Anything else needed to ensure someone could reproduce your calculations (maybe ultimately ask someone who has run Desmond, but not done these calculations, to have a look at it for reproducibility - i.e. while we have the paper with the Schrodinger folks to edit would be a good time for this. Maybe Vickie or Caitlin, I think they've run Desmond before...?)

\section{Results}
\subsection*{Calculated free energies depend strongly on starting protein conformation}
%P1: Open with the main problem in default protocol. 
%Describing the approach in broad terms. 
%Give them the conceptual tool to understand most of the work.
%Drive back the main problem in default protocol.
Using the default FEP/REST methodology \cite{FEP/REST}, we find calculated free energies significantly depend on the protein starting conformation, especially for large perturbations (i.e. opening the cavity from the closed state).
To illustrate this, we begin our molecular dynamics simulations both from the protein closed and open conformations then perform alchemical transformations to ligands that occupy the opposite protein conformational state.
In this study, root-mean-square-deviation (RMSD) of the backbone atoms in the F-helix is used to determine the conformational state of the protein over the course of the simulation.
Here, we demonstrate the default 5ns simulation time and REST region selection in the Schr\"{o}dinger FEP workflow are insufficient for adequate sampling of the motion in the F-helix and does not eliminate the dependence on the initial protein state.

%P2: Discuss closed-open case to show the problem in default protocol
An examination of the largest alchemical transformation, benzene to n-hexylbenzene, highlights the sampling problems faced when using the standard FEP/REST protocol.
From experimental data of ligand occupancies (Table~\ref{tbl:loop-occ}), we expect in our simulations of n-hexylbenzene to see the protein primarily in the open state over the closed state.
Instead, we find the protein remains trapped in its initial conformational state whether we start from closed (Fig~\ref{fig:c_opls3_1/RMSD-replica11}) or open (Fig~\ref{fig:o_opls3_1/RMSD-replica11}) over the course of the 5ns simulation.
From the protein closed simulation, the protein only begins to enter the intermediate state around 3ns but never enters the open conformation.
As the protein tries to accommodate n-hexylbenzene and enter its preferred open state, protein-ligand strain results, yielding a positive value for $\Delta\Delta G_{calc}$(+4.13 kcal/mol).
On the other hand, in the protein open simulations, the protein already begins in its preferred state for n-hexylbenzene and stays only in this open state.
As expected, the $\Delta\Delta G_{calc}$ comes out negative(-0.61 kcal/mol) as there is no occurrence of large protein-ligand strain in order to open the cavity.
Ultimately, we arrive at two very different relative free energies values, where the discrepancy is as large as +4.74 kcal/mol for the same transformation of benzene to n-hexylbenzene\footnotemark.
Collectively, when we view the discrepancy of all calculations involving closed-open transformations we find the root-mean-square-inconsistency (RMSI) to be +4 kcal/mol (Table~\ref{tbl:C-O}).  
Clearly, despite the use of FEP/REST, we are unable to adequately sample all the relevant states within the standard 5ns time frame, resulting in such large differences in $\Delta\Delta G_{calc}$.

\footnotetext{It should be noted that the binding affinities of n-pentyl/n-hexylbenzene to T4-L99A are not known and were inaccessible in experimental studies due to solubility limits. \cite{Merski2015}
For cases involving these ligands, we only focus on the convergence of the calculated free energies between simulations starting from protein closed or open.}

%P3: Discuss close-int case to show there is still a dependence even for small changes
In the case of more moderate alchemical transformations, such as cases that involve the set of closed ligands (i.e. benzene to n-propylbenzene) to intermediate ligands (i.e. sec-/n-butylbenzene), we find that the calculated free energies still have some (albeit much smaller) dependence on the initial protein conformation using the default protocol.
For the set of transformations to the intermediate state, the RMSI in $\Delta\Delta G_{calc}$ for protein closed versus open simulations is +0.60 kcal/mol (Table~\ref{tbl:C-I}). 
However, when we compare $\Delta\Delta G_{calc}$ against $\Delta\Delta G_{exp}$ for transformations involving n-butylbenzene, we find that simulations starting from the protein closed conformation are further from converging to $\Delta\Delta G_{exp}$ than when starting from the protein open conformation.
From the protein closed simulations, the RMS error relative to $\Delta\Delta G_{exp}$ is +1.40 kcal/mol (Table~\ref{tbl:C-nbutyl_closed}), while for the protein open simulations it is +0.70 kcal/mol (Table~\ref{tbl:C-nbutyl_open}).
Based on experimental evidence (Table~\ref{tbl:loop-occ}), we should expect to see some sampling (~30\%) of the open conformation for the n-butylbenzene ligand if the calculations are converged.
Evidently, we find the simulations remain trapped in their respective starting conformations, resulting in inadequate sampling in the protein closed simulations (Fig.~\ref{fig:c_exp_opls3_11/RMSD-replica11}) versus the protein open simulations (Fig.~\ref{fig:o_exp_opls3_24/RMSD-replica11}).
Despite performing much smaller alchemical transformations, we still encounter sampling problems that result in $\Delta\Delta G_{calc}$ that depend on the intial protein conformation, evident when comparing to $\Delta\Delta G_{exp}$.

\subsection*{Including protein residues into the REST region (pREST) improves sampling}
%P1: Recap problem, summarize solution and its effect
Primarily, we encounter major sampling problems when we begin our simulations from the protein closed state and attempt a mutation which should result in the helix opening.
In order to facilitate protein motion, we included 3 key residues spanning the F-helix region into the REST region, which we will denote simulations using this with pREST. %(see Methods for details?)
By expanding the REST region, we are able to drive the F-helix out its initial state trap faster by locally heating up key regions and thereby reduce our problem with inadequate sampling.

%P2 REST improvement qualitatively
To demonstrate the REST improvement over the default protocol, we return to the case of benzene to n-hexylbenzene.
Here, we show the facilitation of the helix motion by first referring to Fig~\ref{fig:c_opls3_1/RMSD-replica11} which shows that there is no sampling of the open state for the default protocol.
On the other hand with the pREST, we see a few open state points around 3ns and even a single open point before closing again in our initial step. (Fig~\ref{fig:c_opls3_rest1_1/RMSD-replica11}.
Alternatively, we can further illustrate the enhancement of protein transitions by viewing all replicas collectively. 
In Fig~\ref{fig:c_opls3_1/colormap} and Fig~\ref{fig:c_opls3_rest1_1/colormap}, we perform the same RMSD analysis but instead represent each time point as a colored bar and no longer plot the raw RMSD.
Visually, it is easy to see that there are far less transitions in default simulations (Fig~\ref{fig:c_opls3_1/colormap}) as opposed to pREST simulations (Fig~\ref{fig:c_opls3_rest1_1/colormap}).
%DLM: Nice graphs! Will need axis/color labels though. But these are very helpful.
 
%P3 REST improvement quantitatively in C-O and C-I sets
Collectively, for all our closed/open and closed/intermediate transformations using pREST, we find only some minor improvements in the RMSI and even cases where we perform worse.
For closed/open transformations (Table~\ref{tbl:C-O_pREST}), the RMSI improves to +2.78 kcal/mol (previously +4 kcal/mol) but gets regresses slightly to +0.78 kcal/mol (previously +0.60 kcal/mol) for closed/intermediate cases (Table~\ref{tbl:C-I_pREST}).
In general, simulations starting from the closed state had $\Delta\Delta G_{calc}$ values that moved towards favorability (i.e. more negative $\Delta\Delta G_{calc}$) and while protein open simulations $\Delta\Delta G_{calc}$ values tended towards unfavorability (i.e. more positive $\Delta\Delta G_{calc}$).
This is indicative of the fact that pREST is indeed improving sampling, but it is evident that our $\Delta\Delta G_{calc}$ are far from convergence given the RMSI is still large, especially for closed/open transformations.

\subsection*{Long simulations enhances protein conformational sampling from more exchanges}
%P1 Recap problem, summarize solution and effect
Although we see improvements in sampling with pREST, the standard implemented time frame of 5ns clearly is not long enough completely capture the transition of closed to open in the helix.
Here, we simulate 55ns for closed/open transformations and take the final 15ns of the simulation while for closed/intermediate we simulate 25ns and take the last 10ns of the simulation, discarding the initial time as additional equilibration time.
In simply running longer, we allow our simulations to perform more exchanges across replicas and thereby allow for better sampling of all conformational states.

%P2 Longer simulation improvements qualitatively
Returning to our most extreme transformation, benzene to n-hexylbenzene, we have shown pREST alone does not allow for adequate sampling of the open state (Fig~\ref{fig:c_opls3_rest1_1/RMSD-replica11}).
Now, when we run much longer we see far more sampling of the open protein conformational state in the final 10ns window (Fig~\ref{fig:c_opls3_rest1_1/45-55ns/RMSD-replica11}).
In viewing all the replicas (Fig~\ref{fig:c_opls3_rest1_1/cmap-45-55ns}), we illustrate the dramatic increase in protein conformational sampling in stark contrast to our 5ns simulations (Fig~\ref{fig:c_opls3_rest1_1/colormap}). 

%P3 Longer simulation improvements quantitatively
By simulating longer with pREST we dramatically increase our sampling of the intermediate/open states and almost entirely eliminate the dependence on the initial protein conformational state.
For the set of closed/open transformations the RMSI dramatically falls to +0.57 kcal/mol (Table ~\ref{tbl:C-O_pREST-40-55ns}).
In the set of closed/intermediate transformations, only a few cases required an extended simulation time of 25ns, after doing so we obtain an overall RMSI of +0.42 kcal/mol (Table ~\ref{tbl:C-I_pRESText}). 
There still remains some discrepancy between $\Delta\Delta G_{calc}$ from protein open or closed simulations, but it now falls within a much more reasonable range of less than +1 kcal/mol. 
 
\section{Discussion}
%P1: Restate problem: Convergence issues from protein conformational changes
%Even small rearrangements present sampling challenges which impact FE accuracy
%Magnitude/Impact of initial protein conformation
In this study, we find that relative free energy calculations can suffer from substantial convergence problems resulting from relatively modest protein conformational changes.
Although, the protein conformational changes in T4 lysozyme (L99A) are extremely localized to a rearrangement of a single helix, we still encounter sampling challenges.
These problems have profound implications for the accuracy of computed relative free energies in these cases.
Particularly, we find that calculated relative free energies can depend on the initial protein conformational state by up to 4 kcal/mol.

%P2: Restate the experiment carried out to remind the reader.
%Remind main analysis technique and the main idea we can draw from it
By looking at cases that involve a conformation change in the protein, we show the $\Delta\Delta G_{calc}$ is sensitive to the initial protein conformational state.
Such cases are when the alchemical transformation involves mutating ligands that primarily occupy the closed state (i.e. benzene to n-propyl) into ligands that occupy the intermediate (i.e. sec-/n-butyl) or open states (n-pentyl/n-hexyl) (Table~\ref{tbl:loop-occ}).
In tracking the RMSD relative to the crystallographic protein closed structure, we show the protein remains trapped in its initial state throughout the simulation when using the implemented default protocol.
Through remaining trapped, we are unable to adequately sample the correct protein-ligand conformational states and thereby poorly computed $\Delta\Delta G_{calc}$. 

%P3: Broader implications of this study
%Restate general trends from protein closed vs open
%Explicitly state the implications
Without prior knowledge of preferred protein conformational states on ligand binding, we can arrive at very different binding affinity predictions that are sensitive to the initial protein state.
Generally, from protein closed simulations we found $\Delta\Delta G_{calc}$ were highly positive and unfavorable, a result from strain in the protein as the ligand begins to grow in the binding cavity.
However, in protein open simulations we tended to see $\Delta\Delta G_{calc}$ that were negative and favorable as no protein-ligand strain is encountered (Table~\ref{tbl:C-O}).
If we only had the crystal structure of the closed protein-ligand complexes, we would blindly conclude that the much larger, open ligands bind worse than smaller ones.
On the other hand, if only the open protein-ligand complexes were available, the opposite would be concluded in that larger ligands are better binders than smaller ligands.

%P4: Restate the improvements from pREST
%Show where the improvement stems from
%Drive home main point again.
By including key residues into the REST region and simulating longer, we reduce the $\Delta\Delta G_{calc}$ dependence on the initial protein conformation to a more reasonable range of less than 1 kcal/mol.
Through expanding the REST region, intermediate lambda windows are able to more easily access the intermediate and open conformations by effectively heating key residues that would facilitate protein motion, illustrated in Fig~\ref{fig:c_opls3_rest1_1/colormap}.
Further, by simulating longer we allow for more exchanges between replicas, which in turn enhances sampling at our physically relevant end state lambda windows (Fig~\ref{fig:c_opls3_rest1_1/cmap-45-55ns}).   
With these modifications to the default protocol, we almost completely converge our $\Delta\Delta G_{calc}$ to the same value regardless of the starting protein conformation.

%P5: Closing important thoughts on discussion followed by speculation on best practices for other future studies
From our results, we present strong evidence for the importance of studying kinetically distinct protein conformational states prior to performing binding free energy calculations.
Without performing such initial studies, early drug discovery projects are faced with the hazard that the effect of the initial protein structure on predicted binding free energies is essentially hidden.
Only from prior crystallography studies\cite{Merski2015} and our systematic trials of varying the protein starting structure were we able to identify this problem and correct it.
Generally, our brute-force approach of simulating longer and multiple trials with varied protein structures is not a desirable or a even a feasible approach, especially in early drug discovery phases.
At the industrial level, ligand libraries can be large---driving computational cost exponentially if we simulate longer---or experimental structures can be sparse for new therapeutic protein targets.
For future studies, approaches using Markov State Models (MSMs)\cite{MSM} can potentially be of great use for identifying discrete protein conformations.
MSMs build a representation of the conformational space from batches of short molecular simulations, whereby the discrete states and transition rates between them can be determined.
Utilizing MSMs can thereby provide useful insight on the various protein conformational states before running free energy predictions.

\section{Conclusions}
%P1: Broadly state the importance of knowing the various protein conformational states
Overall, we have shown that the presence of kinetically distinct protein conformational states---that are modulated by ligand binding---can dramatically impact the accuracy free energy calculations.
Had we not known of the various protein-ligand states, we would not have been able to see our predictions had a strong dependence on the protein-ligand structure we had started our simulations from.
In this study, we would have encountered a worst-case scenario if we only had the structure of the apo protein available, as is often the case in early stages of drug discovery.
Then, a medicinal chemist could be tasked with docking a library of ligands and running binding free energy predictions in order to determine the most suitable candidates to push forward.
Dangerously, the chemist would then discard ligands with apparently low binding affinities, where these ligands only appear unfavorable because of unsampled protein conformational changes.
Without prior knowledge of the existence of other protein conformational states, one would never know that these calculations are actually incorrect due to poor convergence from insufficient sampling.

%P2: Broadly state the problem of protein sampling, how we resolved it, and the importance of this study.
Although FEP calculations have shown tremendous recent successes \cite{FEPplus}, we are still faced with challenges in adequate protein sampling.
Using T4 lysozyme (L99A) as our model system, we highlight sampling problems even from a relatively small (~1-3\AA) and localized single helix rearrangement in response to a series of growing ligands.
In this study, we show that using a typical 5ns simulation time with only ligand atoms in the REST region leads to free energies that significantly depend on the initial protein conformational state. 
Only by longer simulation times and expansion of the REST region to include key protein residues were we able to eliminate the dependence on the starting conformation and come close to convergence.
More importantly, this demonstrates that special attention and care should be exercised when performing binding affinity predictions where regions of flexibility surround the binding site. 

\pagebreak
\section*{Figures}
\subfile{T4figs}

\clearpage
\section*{Tables}
\subfile{T4tables}

\clearpage
\bibliography{references}

\pagebreak
\subfile{T4supp}

\end{document}
